% !TEX root = base-r.tex

%%% This is all first draft/getting ideas into place. -- Jan
\begin{block}{What is Transparency}
   \begin{quote}
       Having one\textquotesingle s actions open and accessible for external evaluation. Transparency pertains to researchers being honest about theoretical, methodological, and analytical decisions made throughout the research cycle.
   \end{quote} 
   Framework for Open and Reproducible Research Training (\href{https://forrt.org/glossary/transparency/}{FORRT.org})  
  
  \blocksubtitle{Why be transparent?}
  \begin{itemize}
  \item Help readers and reviewers understand your work
  \item Helps you stay on top of your work
  \item Prevent mistakes
  \item Work faster
  \item Create better research. 
  \item Increase citations and promote reuse
  \end{itemize}  
  Being transparent benefits you as much as others~\footnotemark[1]. 
  %\cite{markowetz_five_2015}:
  
%   \begin{itemize}
%       \item Work aimed to be made transparent has to be kept orderly and readable. This is easier for everyone (yourself and others) to follow and understand. 
%       \item Well kept materials are more easily reused by others and yourself
%       \item Transparent materials and data allow reviewers to more easily follow your argument
%       \item Transparent data/research materials can be checked more easily for mistakes, making it easier to catch them before publication
%   \end{itemize}


  %\blocksubtitle{Is transparent research inherently better?}
  %\begin{itemize}
   %   \item Not necessarily, \textbf{but} 
      %\item going through pre-registration forces you to think about your study before you commit 
     % \item keeping track of your work prevents mistakes
     % \item making data and materials available increase citation counts
      
      
 %     (e.g., what is my dependent variable, how do I measure it). This can help you keep track of your works intended core idea and save you from uncomfortable surprises at a point where you can no longer change certain things (e.g., after data collection).
 % \end{itemize}
 
% \blocksubtitle{\faIcon{skull-crossbones}  What do we want to avoid? \faIcon{skull-crossbones}}
  
%   \begin{itemize}
%       \item Questionable Research Practices (QRP) e.g.,
%       \item HARKing (Hypothesizing After Result Known)
%       \item p-hacking 
%       \item cherry-picking
%       \item data dredging
%   \end{itemize}

 
%  \blocksubtitle{How to be Transparent without (much) Extra Work?}
%  \begin{itemize}
%    \item Tacking on open science practices at the end of a project %often feels like extra work
%    \item Integrate transparency from day one of your research (check the timeline on the back).
%  \end{itemize}
%   Open Science practices can seem like extra work, particularly if you when having to implement them into an existing project. However, when you are intending to be transparent at publication from the start, the required work can be (almost) seemingly integrated into your normal work. Subsequently it does not require much more time or effort during any single step of research.
  
 
  
  \blocksubtitle{What should I make transparent?}
  Almost any researcher can integrate transparent practices. However:
\begin{itemize}
      \item Keep in mind participant safety and rights 
      \item Some data cannot be shared safely
      \item Use data availability statement to report what can and can't be shared
  \end{itemize} 
Even if data must be kept private, share what materials you can (interview questions, analysis code, or other materials). 
% and use a \href{https://www.cambridge.org/core/services/authors/open-data/data-availability-statements}{data availability statement} to transparently report what you can and can\textquotesingle t share.
\end{block}

\begin{block}{Inspiration/Examples}
\blocksubtitle{Papers}
Broman, Wu (2018). Data organization in spreadsheets \\ \href{https://peerj.com/preprints/3183/}{peerj.com/preprints/3183}
\\ Wickham (2014). Tidy Data. \href{http://www.jstatsoft.org/v59/i10/}{jstatsoft.org/v59/i10}
\\ Wilson, G. et al (2017). Good enough practices in scientific computing \href{https://doi.org/10.1371/journal.pcbi.1005510}{doi.org/10.1371/https://doi.org/10.1371/journal.pcbi.1005510}
\blocksubtitle{Organizations}
 Center for Open Science \href{https://www.cos.io/}{cos.io}
\\ The Alliance for Open Scholarship \href{https://www.all4os.org/}{all4os.org}
\\ Project TIER \href{https://www.projecttier.org/}{projecttier.org}
\\ Framework for Open and Reproducible Research Training \href{https://forrt.org/glossary/transparency/}{FORRT.org}
\\ FOSTER Open Science \href{https://www.fosteropenscience.eu/}{fosteropenscience.eu}

\end{block}

\footnotetext[1]{Markowetz, F. (2015). Five selfish reasons to work reproducibly. Genome biology, 16(1), 1-4.}